\documentclass[10pt]{article}
\usepackage[utf8]{inputenc}
\title{IF793 -Projeto e Implementação de Jogos 2D}
\author{Vinícius Santos Lourenço }
\date{April 2019}
\usepackage[portuguese]{babel}
\usepackage{natbib}
\usepackage{graphicx}

\begin{document}

\maketitle

\section{Introdução}
O que são "jogos" ? Jogos são veículos de treinamento, entretenimento e educação que contém a figura de um jogador. Os jogos despertam o espírito de competição, a socialização, a imaginação, a curiosidade, entre outros sentimentos. Atualmente, o mercado de jogos tem crescido cada vez mais, conquistando ainda mais pessoas, sendo por jogos de console, pc ou mobile.  A disciplina de Projeto e Implementação de jogos 2D\citep{pijcin} ensina os conhecimentos necessários para desenvolver jogos em 2D. Alguns dos temas abordados nesse estudo, são:
\begin{itemize}
\item História e categoria de jogos: Um estudo introdutório sobre o curso.
\item Projeto de jogos :Onde é abordado conceitos como roteiro, interface, etc.
\item Conceitos gráficos :Onde se aborda conceitos como modelo 2D, sprites, modelagem e animação 3D.
\item Outros conceitos como: Gráficos, sons, redes, inteligência artificial
\item Entre outros tópicos de interesse
\end{itemize}
\begin{figure}[h!]
\centering
\includegraphics[scale=0.17]{pokemon}
\caption{The Game \citep{imagempokemon}}
\label{fig:universe}
\end{figure}

\section{Relevância}
\begin{itemize}
\item Pontos positivos: Essa disciplina proporciona uma visão geral do mercado de games e prepara o aluno para o mercado incentivando o trabalho em equipe.
\item Pontos negativos: A dependência  com as outras áreas, acaba por desfavorecer a prática individual, que apesar de ser menos requisitada pode ser vista como um "isolamento" por aqueles que não se habituaram.
\end{itemize}
\section{Relação com outras Disciplinas}



% ######## init table ########
\begin{table}[h]
 \centering
% distancia entre a linha e o texto
 {\renewcommand\arraystretch{1.25}
 \caption{A sample table}
 \begin{tabular}{ l l }
  \cline{1-1}\cline{2-2}  
    \multicolumn{1}{|p{3.850cm}|}{Disciplina \centering } &
    \multicolumn{1}{p{4.217cm}|}{Relação \centering }
  \\  
  \cline{1-1}\cline{2-2}  
    \multicolumn{1}{|p{3.850cm}|}{IF680-Processamento Gráfico\citep{pgcin}} &
    \multicolumn{1}{p{4.217cm}|}{A área de manipulação gráfica é muito importante no desenvolvimento de jogos onde usa-se os conhecimentos de processamento gráfico na programação e produção de um motor gráfico de um game.}
  \\  
  \cline{1-1}\cline{2-2}  
    \multicolumn{1}{|p{3.850cm}|}{IF687-Introdução à Multimídia \citep{imcin}} &
    \multicolumn{1}{p{4.217cm}|}{Essa área é muito importante no desenvolvimento de jogos pois cuida dos conceitos de sons, animações e vídeos no computador. }
  \\  
  \hline

 \end{tabular} }
\end{table}
\bibliographystyle{plain}
\bibliography{vsl}
\end{document}

 