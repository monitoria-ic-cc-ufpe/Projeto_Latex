\documentclass{article}
\usepackage[utf8]{inputenc}
\usepackage[portuguese]{babel}

\title{Introdução à Biologia Molecular}
\author{Pedro Henrique Souza Balbino}
\date{May 2019}

\usepackage{natbib}
\usepackage{graphicx}
\usepackage{longtable}

\begin{document}

\maketitle

\centering
\section{Introdução}
A cadeira de Introdução à Biologia Molecular (IF803) é uma cadeira eletiva lessionada dentro do perfil Bio-informática da graduação em Ciência da Computação. A cadeira visa a coompreensção holistica da biologia molecular dentro da área de computação.
\includegraphics[scale= 0.5]{Imagem.jpg} \centering \citep{desconhecido_2005}

\section{Livros Indicados}
\begin{itemize}
    \item Handbook of Computational Molecular Biology.; \citep{aluru2005handbook}
    \item An Introduction to Bioinformatics Algorithms; \citep{jones2004introduction}
    \item Computational Molecular Biology - An Algorithmic Approach; \citep{pevzner2000computational}
\end{itemize}

\section{Relevância}
Tendo ocorrido entre 1990 e 2003, o projeto genoma humano cataputou a relevancia e cobertura do estudo de biologia da computação, uma vez que tais feitos não teriam sido possiveis sem o uso te algoritimos extremamentes sfisticados para epoca. \hfill \break
Tais estudos se fazem vitais por diversas razoes:

\begin{itemize}
    \item Sua importancia no descobrimento e tratamento de novas doenças genneticas;
    \item Relação con metodos de transgenia, usados tanto para o desenvolvimento de farmacos quanto para outros fins;
    \item Amplo campo de relações com o proprio desenvolvimento da informatica como um campo de estudo.
\end{itemize}

\section{Relação com outras disciplinas}

\begin{table}[h]
\centering
\begin{tabular}{cllll}
\cline{1-2}
\multicolumn{1}{|c|}{Códigos} & \multicolumn{1}{c|}{Relações} &  &  &  \\ \cline{1-2}
\multicolumn{1}{|c|}{IF803} & \multicolumn{1}{l|}{\begin{tabular}[c]{@{}l@{}}"O objetivo deste curso é apresentar a área de Biologia Molecular \\ Computacional, introduzindo conceitos essencias da Biologia para \\ a compreensão da área de aplicação, dos problemas práticos de \\ Bio-Informática e Biologia Computacional que envolve a manipulação \\ e análise de dados biológicos e problemas da área, juntamente com \\ abordagens computacionais para a sua solução."- Segundo o site da cadeira\end{tabular}} &  &  &  \\ \cline{1-2}
\multicolumn{1}{|c|}{IF806} & \multicolumn{1}{l|}{\begin{tabular}[c]{@{}l@{}}"Apresentar problemas contemporâneos para os quais o aluno possa conhecer a \\ literatura associada e colocar a mão na massa, utilizando recursos da Computação, \\ como técnicas de Aprendizagem de Máquina, Algoritmos combinatoriais, Métodos \\ estatísticos, ou simulação. Esta disciplina pode servir como base para um TG ou \\ futuro mestrado nesta área, oferecendo a oportunidade do aluno se familiarizar \\ com a pesquisa corrente." -Segundo o site CinWiki\end{tabular}} &  &  &  \\ \cline{1-2}
\multicolumn{1}{l}{} &  &  &  & 
\end{tabular}
\end{table}


\bibliographystyle{plain}
\bibliography{phsb}
\end{document}
