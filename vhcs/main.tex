\documentclass[10pt]{article}
\usepackage[utf8]{inputenc}
\usepackage[portuguese]{babel}
\title{Dark Souls}
\title{IF708 - Programação Funcional}
\author{Vinícius Henrique}
\usepackage[normalem]{ulem}
\useunder{\uline}{\ul}{}
\usepackage{natbib}
\usepackage{graphicx}
\usepackage{indentfirst}
\usepackage[normalem]{ulem}
\useunder{\uline}{\ul}{}

\begin{document}

\maketitle

\section{Introdução}
A disciplina de "Programação Funcional"\citep{r1} foca em aprofundar os conceitos iniciais de programação, que são abordados desde o início do curso, como em Introdução a Programação\citep{r2} e em Paradigmas de Linguagens de Programação\citep{r3}.

A mesma aborda, dentre outros tópicos\citep{r1}:
\begin{itemize}
    \item \textbf{Programação:} Programação com funções, programação com listas, recursão;
    \item \textbf{Estrutura de Dados:} Tipos de dados algébricos, árvores;
    \item \textbf{Outros:} Prova de propriedades sobre programas, provas por indução, avaliação estrita e preguiçosa.
\end{itemize}

A linguagem usada para trabalhar é Haskell\citep{r4}, que é uma boa linguagem para se aprender programação funcional.\citep{r5}

Como se demonstra com essa descrição, a disciplina pertence à grande área de algoritmos, na computação.

\begin{figure}[h!]
\includegraphics[scale=0.7]{bigstock-Functional-Programming-Code-231296137-1184x978.jpg}
\centering
\label{fig:i1}
\caption{Programação Funcional}
\end{figure}

\section{Relevância}
O uso da matemática aprendida nos estágios iniciais do curso é reforçado e colocado em prática na disciplina, e o raciocínio e desenvolturas aqui presentes auxiliam o aluno grandemente em seu desenvolvimento.
\section{Relação com Outras Disciplinas}
\begin{table}[h!]
\begin{tabular}{|c|c|}
\hline
\begin{tabular}[c]{@{}c@{}}Introdução a Programação\\ (IF669)\end{tabular}                & \begin{tabular}[c]{@{}c@{}}Inicia o aluno nos conceitos iniciais de\\ programação, indispensáveis na disciplina.\end{tabular}                                       \\ \hline
\begin{tabular}[c]{@{}c@{}}Paradigmas em Linguagens\\ de Programação (IF686)\end{tabular} & \begin{tabular}[c]{@{}c@{}}Continua o desenvolvimento dos conceitos\\ apresentados na IF669 e abrangem parte \\ dos tratados na Programação Funcional.\end{tabular} \\ \hline
{\ul Matemática Discreta (IF670)}                                                         & \begin{tabular}[c]{@{}c@{}}Discute estruturas matemáticas e provas\\ importantes para a disciplina.\end{tabular}                                                    \\ \hline
\end{tabular}
\end{table}
\bibliographystyle{plain}
\bibliography{references}
\end{document}

