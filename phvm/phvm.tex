\documentclass[10pt]{article}
\usepackage[utf8]{inputenc}
\usepackage{float}
\usepackage{url}
\usepackage[portuguese]{babel}
\usepackage{graphicx}
\usepackage{microtype}
\usepackage{hyperref}
\usepackage[T1]{fontenc}

\title{IF699 - Aprendizagem de máquina}
\author{Pedro Moura}
\date{\vspace{-5ex}}

\usepackage{natbib}

\begin{document}

\maketitle

\section{Introdução}

A disciplina de aprendizagem de máquina tem o objetivo de fazer com que os discentes estudem técnicas computacionais que apresentem características de aprendizagem automática. Nela também é fornecido uma visão geral da área de aprendizagem de máquina, estuda métodos e técnicas de aprendizagem de máquina, bem como aspectos teóricos e práticos da aprendizagem de máquina.

\begin{itemize}
  \item Seus objetivos são fazer com que o discente conheçam técnicas e ferramentas de
\\- Aprendizado supervisionado
\\ - Aprendizado não supervisionado
\\ - Desenvolver programas que melhorem seu desempenho por meio de experiências, de grandes volumes de dados e da geração de hipóteses a partir de dados.
\end{itemize}

A disciplina está inserida na área de SISTEMAS INTELIGÊNTES, sendo o profissional dessa área responsável pelo desenvolvimento e a criação de programas que melhorem a si mesmos pelo aprendizado através de diferentes métodos.

\begin{figure}[]
    \centering
    \includegraphics[scale=0.3]{aprendizagemdemaquina.jpg}
    \caption{APRENDIZAGEM DE MÁQUINA \cite{quarta}}
    \label{fig:realidadevirtual}
\end{figure}

\section{Relevância}
Essa disciplina é voltada para pessoas com interesse em computação inteligênte, na utilização de grandes volumes de dados e de sistemas inteligêntes para a melhoria de seus softwares e aspiração para a área de inteligência artificial. Ela é um grande complemento para o currículo de um cientista da computação por ser uma das área que mais se desenvolve e mais cresce tanto no âmbito acadêmico quanto comercial.\cite{primeira}

\section{Relações}
Algumas disciplinas que são relacionadas com a aprendizagem de máquina que estão na grade curricular.

\begin{tabular}{lllll}
\cline{1-2}
\multicolumn{1}{|l|}{IF705 - Automação Inteligente \cite{segunda}} & \multicolumn{1}{l|}{\begin{tabular}[c]{@{}l@{}}A automação inteligente estuda \\ como fazer com que máquinas \\ imitem o aprendizado, \\ a tomada de decisões e ações \\ de seres usamos, o qual emprega \\ técnicas  aprofundadas na disciplina \\ de  aprendizagem de máquina.\end{tabular}}                                                            &  &  &  \\ \cline{1-2}
\multicolumn{1}{|l|}{ IF702 - Redes Neurais \cite{terceira}}   & \multicolumn{1}{l|}{\begin{tabular}[c]{@{}l@{}}A disciplina de redes neurais \\ busca a organização de bases \\ de dados através de \\ treinamento prévio do software, \\ o qual usa métodos de aprendizagem \\ de máquina para ser treinado.\end{tabular}} &  &  &  \\ \cline{1-2}
                                                      &                                                                                                                                                                                                                                                                &  &  &  \\
                                                      &                                                                                                                                                                                                                                                                &  &  & 
\end{tabular}




\bibliographystyle{plain}
\bibliography{phvm}

\end{document}