\documentclass{article}
\usepackage[utf8]{inputenc}
\usepackage{indentfirst}
\usepackage{natbib}
\usepackage{graphicx}
\usepackage[portuguese]{babel}

\title{IF 754 - Computação Musical e Processamento de Som}
\author{Mateus Alves da Silva}
\date{}

\begin{document}

\maketitle

\section{Introdução}
Computação musical e processamento de som é uma cadeira que tem como objetivo principal formar alunos capazes de escrever programas de ação multimídia que verdadeiramente adaptem-se aos meios computacionais hoje disponíveis. 

Ela é lecionada pelo professor Geber Ramalho, que é doutor em informática na área de Inteligência Artificial pela universidade de Sobornne, e foi ofertada pela última vez no primeiro semestre de 2016.  \citep{sitedoprof}

Esta disciplina oferece aos alunos a possibilidade de complementar seus conhecimentos relativos à natureza da forma sonora, aos algoritmos para a síntese e processamento de sons digitais, e às técnicas de representação e manipulação de informações  musicais, incluindo wave, MIDI(Musical Instruments Digital Interface), MP3, RealAudio, etc.

\begin{figure}[h!]
\centering
\includegraphics[scale=0.31]{mas8}
\caption{Livro texto da disciplina \citep{livro}}
\label{fig:livrotexto}
\end{figure}

\section{Relevância \citep{infodacadeira}}
A matéria é uma eletiva voltada para  os alunos que tenham interesse na área, não necessariamente habilidades musicais, pois os conhecimentos necessários são ensinados na propria disciplina.

\section{Relação com Outras disciplinas}
Para poder cursar a referida disciplina, o aluno deverá dominar, ou estar em vias de dominar, alguns conhecimentos básicos de computação ensinados nas seguintes disciplinas: Algoritmos e estruturas de dados (pré-requisito) e Linguagens de Programação 3 (co-requisito).
\begin{table}[h]
    \centering
    \begin{tabular}{c|c}
        Disciplina & Relações \\
        \hline
         & Como a cadeira tem por objetivo tornar o discente\\
        Algoritimos e estrutura de dados & hapto para criar programas na area musical,\\
          & o conhecimento em algoritimos é impresindivel.\\
        \hline
        Linguagens de Programação 3 & Análogo à algorítimos\\
        \hline
         & Orquestração MIDI, Reconhecimento de altura e ritmo,\\
        Inteligencia artificial & Previsão de acordes, composição automática, Sistemas\\
         & de acompanhamento automático
        
    \end{tabular}
    \caption{Cadeiras relacionadas \citep{sitedadisciplina}}
    \label{tab:my_label}
\end{table}

\bibliographystyle{plain}
\bibliography{mas8}

\end{document}
