\documentclass{article}
\usepackage[utf8]{inputenc}

\title{IF803 - Introdução à Biologia Molecular}
\author{Daniel de Azevedo Pacheco }
\date{Maio de 2019}
\usepackage[brazil]{babel}

\usepackage{natbib}
\usepackage{graphicx}

\begin{document}

\maketitle

\section{Introdução}
Introdução a biologia molecular é uma disciplina da graduação em Ciência da Computação, que visa apresentar os conceitos essenciais da biologia com o objetivo de propiciar o entendimento dos problemas e aplicações da bioinformática\cite{site}, área de estudo que corresponde à aplicação das técnicas da informática, no sentido de análise da informação, nas áreas de estudo da biologia. Como um campo interdisciplinar da ciência, a bioinformática combina a biologia, ciência da computação, estatística, matemática e engenharia para analisar e interpretar e processar dados biológicos. 
\begin{figure}[h]
    \centering
    \includegraphics[width=10cm]{bio.png}
    \caption{códigos genéticos(material de estudo da bioinformática) \cite{img}}
    \label{fig:my_label}
\end{figure}
\section{Relevância}
Introdução à Biologia Molecular é a primeira disciplina do currículo de ciência da computação da UFPE que abrange a área da bioinformática, o campo interdisciplinar voltado a aplicação de técnicas de informática para a análise de informações biológicas. Surgindo na década de 70 esse campo de estudo nasceu com o foco de estudar a biologia molecular por meio dos computadores, que começavam a ganhar espaço no mundo acadêmico por permitir fazer análise das moléculas biológicas em uma velocidade muito maior que os seres humanos sozinhos\cite{slide1}.
Um dos momentos mais importantes para a bioinformática foi quando os órgãos americanos National Institutes of Health (NIH) e o Department of Energy (DOE) se juntaram para criar o Projeto Genoma Humano, um mapeamento de todo o genoma humano, que foi concluído em 2003.\cite{slide2}
A disciplina tem como objetivo passar para os alunos os primeiros conhecimentos das ciências biológicas que serão necessários para o desenvolvimento de sistemas de análises moleculares e como funcionam esses processos.

\section{Relações interdisciplinares}
\begin{table}[h]
 \centering
 {\renewcommand\arraystretch{1.25}
 \begin{tabular}{ l l }
  \cline{1-1}\cline{2-2}  
    \multicolumn{1}{|p{4.083cm}|}{disciplina} &
    \multicolumn{1}{p{6.050cm}|}{relação}
  \\  
  \cline{1-1}\cline{2-2}  
    \multicolumn{1}{|p{4.083cm}|}{IF804: COMPARAÇÃO ANALISE SEQUENCIAS DNA \centering } &
    \multicolumn{1}{p{6.050cm}|}{Outra cadeira da matriz curricular do curso relacionada a bioinformática que estuda as sequencia de DNA e suas relações. É pre requisito para projeto em bioinformática junto com Introdução a Biologia Molecular}
  \\  
  \cline{1-1}\cline{2-2}  
    \multicolumn{1}{|p{4.083cm}|}{IF805: PROJETO EM BIO INFORMÁTICA \centering } &
    \multicolumn{1}{p{6.050cm}|}{Disciplina que possui Introdução a Biologia Molecular como pre requisito e da continuidade ao estudo da bioinformática o curso. A disciplina propõe para o aluno que defina um problema em bioinformática, estude e proponha uma solução para o mesmo.}
  \\  
  \cline{1-1}\cline{2-2}  
    \multicolumn{1}{|p{4.083cm}|}{IF806: TÓPICOS AVANC.EM BIO INFORMÁTICA. \centering } &
    \multicolumn{1}{p{6.050cm}|}{Outra disciplina que tem como pre requisito Introdução a Biologia Molecular. A cadeira tem como objetivo o estudo das técnicas mais recentes na área da bioinformática permitindo o aluno conhecer o estado atual dessa área de pesquisa.}
  \\  
  \hline
 
 \end{tabular} }
 \caption{\cite{PerfilCurricular}}
\end{table}


\bibliographystyle{plain}
\bibliography{dap5}

\end{document}

