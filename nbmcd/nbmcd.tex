\documentclass[10pt]{article}
\usepackage[utf8]{inputenc}

\title{IF672 - Algoritmos e Estruturas de Dados}
\author{Nilo Bemfica Mineiro Campos Drumond (nbmcd)}
\date{Abril 2019}

\usepackage{natbib}
\usepackage{graphicx}

\begin{document}

\maketitle


\section{Introdução}
Algoritmos são sequências de passos computacionais que recebem um valor de entrada, e retornam outro de saída. Muito importante para os algoritmos são as estruras de dados, que consistem em meios de armazenar e organizar dados com o objetivo de facilitar o acesso e modificações\cite{algo}. Os principais conceitos abordados na cadeira IF672 - Algoritmos e Estruturas de dados são: conceitos básicos de algoritmos e estruturas de dados; estruturas de dados dinâmicas; busca e ordenação em ED lineares; Arvores binárias; busca e ordenação em ED de n dimensões e grafos\cite{cin-if672}. Tanto algortmos como estruturas de dados fazem parte da subárea Computação Básica\cite{subarea}.

\section{Relevância}
Algoritmos e estruturas de dados são essenciais para qualquer área que envolva desenvolvimento de softwares, resolução de problemas computacionais e afins. É indispensável para qualquer cientista da computação, por isso é cadeira base de qualquer curso de ciência da computação e engenharia da computação.

\section{Relação com Outras Disciplinas\cite{cc}}

\begin{tabular}{lll}
Outras cadeiras & Relação com essa cadeira \\ \hline
IF669 - Introdução a Programação          & Conceitos básicos essenciais para algoritmos. \\ \hline
IF768 - Teoria de Grafos                  & Avança no conceito de grafos
\\ \hline
IF775 - Tópicos Avançados em Algoritmos   & Avança no estudo de algoritmos.                                                                                                          \\ \hline
IF766 - Algoritmos de Aproximação         & Avança no estudo de algoritmos.                                                                                                      \\ \hline
IF682 - Engenharia de Software e Sistemas & Utiliza muitos conceitos de algoritmos.                                                                                                      \\
 \end{tabular}
 
\renewcommand\refname{Bibliografia}
\bibliographystyle{unsrturl}
\bibliography{nbmcd}
\end{document}
