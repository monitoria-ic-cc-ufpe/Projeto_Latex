\documentclass[10pt]{article}
\usepackage[utf8]{inputenc}
\usepackage{graphicx}
\usepackage{natbib}
\usepackage{hyperref}
\title{IF675 - Sistemas Digitais}
\author{Matheus Epitacio Barros de Lucena}
\date{April 2019}

\begin{document}

\maketitle

\section{Introdução}
A disciplina Sistemas Digitais é lecionada no segundo período, o objetivo dela é mostrar aos discentes conhecimentos de circuitos lógicos digitais combinacionais e sequenciais. Esses cirtuitos vão desde dispositivos digitais de pequena complexidade, até implementação de circuitos de média complexidade. Dentro da ementa do curso, na UFPE, estão: Algébra Booleana, Circuitos Combinacionais. A nota do curso é dividida da seguinte forma: A prova vale 70\% da nota e o projeto vale 30\% da nota. No total são 2 provas e 2 notas. Também existem 2 miniprovas que cada uma vale 1 ponto extra para cada prova. A disciplina Sistemas Digitais é uma própria área da computação.   %https://homepages.dcc.ufmg.br/~bigonha/Subareas/computacao.html

\begin{figure}[h]
    \centering
\includegraphics[width=8cm]{mebl/LAB1CIRCUIT.png}
    \label{fig:my_label}
    \caption{Representação de um circuito digital \cite{imagem}}
    
    
\end{figure}
\section{Relevância}
A disciplina Sistemas Digitais é relevante pois, como já dito, ela é uma subárea da Computação. A disciplina também desenvolve o raciocínio lógico dentro. Nessa disciplina é o primeiro contato dos alunos de Ciência da Computação com Hardware, pois nela são mostrados os dispositivos que integram os processadores de informação. Por fim, a disciplina também é útil para conhecer as técnicas que desenvolvem sistemas de tratamento de informação.\\
%http://cin.ufpe.br/~if675/arquivos/aulas/2003-2/unidade1/aula_apresentacao.pdf


Pontos positivos:
\begin{itemize}
    \item Aulas práticas
    \item Desenvolvimento de projetos em grupos
    \item Conhecimento em Hardware
    \item É uma disciplina baseada em resolução de problemas reais
\end{itemize}

\section{Relação com outras disciplinas}
\begin{table}[h]
 \centering
 {\renewcommand\arraystretch{1.25}
 \begin{tabular}{ l l }
  \cline{1-1}\cline{2-2}  
    \multicolumn{1}{|p{5cm}|}{\begin{center}IF674
\end{center}  			


\begin{center}Infraestrutura de Hardware
\end{center}} &
    \multicolumn{1}{p{5cm}|}{ A disciplina Sistemas Digitais é pré-requisito da disciplina de Infraestrutura de Hardware, justamente por ensinar os conceitos básicos que são utilizados na disciplina de Infraestrutura de Hardware, além de ser o primeiro contato dos alunos de Ciência da Computação com hardware. }
  \\  
  \hline

 \end{tabular} }
\end{table}
\section{Referências}
A disciplina usa os seguintes livros como referências: Principles of Digital Design \cite{livro1}; Introdução aos Sistemas DigitaiS \cite{livro2}; Contemporary Logic Design \cite{livro3} ;  Introduction to Computer Engeneering - Hardware and Software Design \cite{livro4}; Circuitos Digitais e Microprocessadores \cite{livro5}.

\bibliographystyle{plain}
\bibliography{mebl}
\end{document}
