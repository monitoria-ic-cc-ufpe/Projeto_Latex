\documentclass[10pt]{article}
\usepackage[utf8]{inputenc}

\title{Aprendizagem de Máquina - IF699}
\author{Heitor da Rocha Coimbra}
\date{Abril 2019}

\usepackage{natbib}
\usepackage{graphicx}

\begin{document}

\maketitle

\section{Introdução}
Basicamente, aprendizagem de maquina faz parte da área da inteligência artificial e é o processo de treinamento de um trecho de software, chamado modelo, para fazer úteis previsões utilizando uma base de dados. Esse modelo pode assim prever rótulos e classificações para dados ainda não apresentados ao programa. Nesta disciplina, se usam essas previsões para atuar num produto; por exemplo, o sistema do Youtube que prevê qual o vídeo mais adequado para ser recomendado para um usuário.\newline


Frequentemente, as pessoas falam de aprendizagem de máquina como um campo com dois grandes paradigmas, aprendizagem supervisionada e não-guiado. No entanto, é mais preciso descrever os problemas desse campo como um espectro entre modelos supervisionados e modelos não-guiados. Um exemplo de modelo que se encontra entre os dois extremos é a aprendizagem reenforçada.\newline


Esse campo da aprendizagem estatística, por exemplo, difere dos outros tipos pois não se assimila resultados de previsões passadas nem se classifica nenhum dado. Imagine que você quer ensinar uma maquina a jogar um videogame básico e nunca perder. Você conecta o modelo ao jogo, e treina o modelo para evitar a tela de "game over". Durante o treinamento, o modelo vai aprendendo os controles e a como reagir aos dados coletados em tempo real.
\cite{alpaydin2015learning} 

\begin{figure}[ht]
\centering
\includegraphics[scale=0.15]{retr.jpg}
\caption{ \cite{photo2012learning}}
\label{fig:my_label}
\end{figure}

\section{Relevância}
Primeiramente, na era da informação, os dados e como intepretá-los é um campo de atuação fundamental para qualquer empresa moderna. Apesar disso, a anáslise de dados tem sido tradicionalmente caracterizada como uma espécie de tentativa-e-erro, já que muitas vezes as técninas de visualização e elementos estatísticos não serve para casos de dados massivos e redundancia estatísticas.\newline


A área da aprendizagem de máquina vem como uma solução a esse caos estabelecido proponto intelignetes alternativas para analisar volumes de dados enormes. É um salto no cmapo da ciência da computação e estatística, auxiliando e criando emergentes aplicações no mercado. A aprendizagem de máquina é capaz de produzir resultados e analises ao desenvolver algoritmos rapidos e modelos matemáticos recheados de algebra linear e cálculo diferencial.
\cite{tom1997learning}

\section{Relação com outras disciplinas}

\begin{table}[h!]
  \begin{center}
    \caption{Interdisciplinaridade \cite{yaser2012learning}}
    \label{tab:table1}
     \begin{tabular}{|l|p{0.28\linewidth}|}
      \textbf{Disciplina} & \textbf{Relação}\\
      \hline
       ET586 - ESTATIST PROBABIL COMPUTACAO & Análise de dados e otimização\newline  \\\hline
       IF680 - PROCESSAMENTO GRAFICO  & Indispensável para Visão  Computacional \\\hline
       IF702 - REDES NEURAIS & Fundamental para vários algorítmos de Aprendizagem de Máquina  \\\hline
       IF704 - PROCESSAMENTO LING. NATURAL & Usa a aprendizagem reforçada para ensinar línguas a máquinas  \\\hline
       IF796 - MINERACAO DA WEB & Possibilita achar, limpar e trabalhar com dados enormes
    \end{tabular}
  \end{center}
\end{table}

\newpage
\bibliographystyle{unsrt}
\bibliography{hrc}
\end{document}
