\documentclass[10pt]{article}
\usepackage[utf8]{inputenc}

\title{Aprendizagem de Máquina - IF699}
\author{Heitor da Rocha Coimbra}
\date{Abril 2019}

\usepackage{natbib}
\usepackage{graphicx}

\begin{document}

\maketitle

\section{Introdução}
Basicamente, aprendizagem de máquina faz parte da área da inteligência artificial e é o processo de treinamento de um trecho de software, chamado modelo, para fazer úteis previsões utilizando uma base de dados. Esse modelo pode assim prever rótulos e classificações para dados ainda não apresentados ao programa. Nesta disciplina, se usam essas previsões para atuar num produto; por exemplo, o sistema do Youtube que prevê qual o vídeo mais adequado para ser recomendado para um usuário.\newline


Frequentemente, as pessoas falam de aprendizagem de máquina como um campo com dois grandes paradigmas, aprendizagem supervisionada e avulsa(sem supervisão).No entanto, é mais preciso descrever os problemas desse campo como um espectro entre modelos supervisionados e modelos avulsos. Um exemplo de modelo que se encontra entre os dois extremos é a aprendizagem por reforço.\newline

\cite{alpaydin2015learning} 

\begin{figure}[ht]
\centering
\includegraphics[scale=0.15]{retr.jpg}
\caption{ \cite{photo2012learning}}
\label{fig:my_label}
\end{figure}

\section{Relevância}
Primeiramente, na era da informação, os dados e como interpretá-los é um campo de atuação fundamental para qualquer empresa moderna. Apesar disso, a análise de dados tem sido tradicionalmente caracterizada como uma espécie de tentativa-e-erro, já que muitas vezes as técnicas de visualização e elementos estatísticos não servem para casos de dados massivos e redundâncias estatísticas.\newline


A área da aprendizagem de máquina vem como uma solução a esse caos estabelecido propondo inteligentes alternativas para analisar volumes de dados enormes. É um salto no campo da ciência da computação e estatística, auxiliando e criando emergentes aplicações no mercado. A aprendizagem de máquina é capaz de produzir resultados e analises ao desenvolver algoritmos rápidos e modelos matemáticos recheados de álgebra linear e cálculo diferencial.
\cite{tom1997learning}

\section{Relação com outras disciplinas}

\begin{table}[h!]
  \begin{center}
    \label{tab:table1}
     \begin{tabular}{|l|p{0.28\linewidth}|}
      \textbf{Disciplina} & \textbf{Relação}\\
      \hline
       ET586 - ESTATIST PROBABIL COMPUTACAO & Análise de dados e otimização\newline  \\\hline
       IF680 - PROCESSAMENTO GRAFICO  & Indispensável para Visão  Computacional \\\hline
       IF702 - REDES NEURAIS & Fundamental para vários algorítmos de Aprendizagem de Máquina  \\\hline
       IF704 - PROCESSAMENTO LING. NATURAL & Usa a aprendizagem reforçada para ensinar línguas a máquinas  \\\hline
       IF796 - MINERACAO DA WEB & Possibilita achar, limpar e trabalhar com dados enormes
    \end{tabular}
    \caption{Interdisciplinaridade \cite{yaser2012learning}}
  \end{center}
\end{table}

\bibliographystyle{unsrt}
\bibliography{hrc}
\end{document}
