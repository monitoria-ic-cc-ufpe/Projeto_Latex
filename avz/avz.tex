\documentclass{article}
\usepackage[utf8]{inputenc}
\usepackage[brazil]{babel}
\title{IF687 -Introdução à Multimídia}
\author{Alexandra Zarzar}
\date{Maio 2019}

\usepackage{natbib}
\usepackage{graphicx}

\begin{document}

\maketitle

\section{Introdução}
O curso de Introdução à Multimídia tem o objetivo de desenvolver as capacidades de avaliar, criar e especificar os componentes multimídia presentes nos mundos virtuais, bem como o estudos deles para a aplicação em diversas áreas, sendo essencial para o grande ramo da computação refrente à mídias e interfaces.  Tal disciplina cobre os tópicos de :
\begin{itemize}
    \item Realidade Virtual, que é responsável pela criação de ambientes novos, independentes do mundo real;
    \item Realidade Aumentada, que diz respeito à interação com o real através das mídias;
    \item Interface, Interação e Navegação, responsáveis pela otimização da relação do usuário com o sistema;
    \item Processamento Gráfico, que explora o importante setor visual da multimídia.
\end{itemize}

\begin{figure}[h!]
\centering
\includegraphics[width=50mm]{imagem-latex}
\caption{Representação de Realidade Virtual}
\label{fig:Realidade Virtual}
\end{figure}

\section{Relevância}
Com a crescente influência dos mundos virtuais na tecnologia somada à baixa quantia de material em português sobre o assunto, sentiu-se a necessidade da criação de um curso sobre essa ramificação da computação. A importância dessa disciplina está baseada nas suas aplicações, sendo a multimídia vastamente explorada nos campos da educação, medicina e entretenimento, por exemplo. Assim, é de grande relevância na grade curricular, já que esse estudo é essencial para a formação dos novos profissionais da área computacional.

\section{Relação com outras disciplinas}


\vspace{0.3cm}
\begin{tabular}{|p{3.0cm}|p{7.0cm}|}
\hline
Disciplinas & Relação\\
\hline
IF669 - Introdução â programação & É necessário que se tenha conhecimento sobre programação para atuar no campo dos mundos virtuais e na construção de seus softwares\\
\hline 
MA531 - Álgebra Vetorial Linear para Computação & Esse curso é essencial para futuros aprofundamentos no campo da computação gráfica, relacionada à parte visual da multimídia \\
\hline 
IF681 - Interfaces Usuário-máquina & Apresenta conceitos básico de Design de interação e Design Thinking para a concepção de sistemas computacionais interativos  \\
\hline 
IF680 - Processamento Gráfico & A disciplina de processamento gráfico aborda a área da computação que estuda assuntos diretamente relacionados ao sentido da visão, essenciais à criação dos mundos virtuais\\
\hline 
IF754 - Computação Musical & Relativo ao desenvolvimento do conhecimento sobre a natureza da forma sonora, os algoritmos para a síntese e processamento 
de sons digitais utilizados no mundo da multimídia\\
\hline

\end{tabular}

\bibliographystyle{plain}
\bibliography{avz}
\nocite{*}
\end{document}
