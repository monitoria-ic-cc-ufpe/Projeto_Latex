\documentclass[10pt]{article}
\usepackage[utf8]{inputenc}

\title{IF681 - Interfaces Usuário-Máquina}
\author{Lázaro Vitor da Silva }
\date{April 2019}


\usepackage{natbib}
\usepackage{graphicx}
\usepackage[T1]{fontenc}
\usepackage{float}
\usepackage{url}
\usepackage[brazil]{babel}
\usepackage[table,xcdraw]{xcolor}

\begin{document}

\maketitle

\section{Introdução}
\paragraph{}A disciplina de Interfaces Usuário-Máquina compreende o escopo de conhecer as necessidades de um público e idealizar uma solução para a necessidade encontrada, conversando com pessoas que conhecem e/ou vivenciam o tema, para que a solução desenvolvida seja pertinente. \cite{pagina} Após a cadeira, os alunos serão capazes de:\newline
\textbf{Compreender os problemas do usuário e resolvelos:} com ajuda de um público alvo, conseguimos ter uma boa ideia do que se passa e como resolver tal problema e para isso, procuramos utilizar o modo mais rápido, prático e que a solução sirva para o dado problema.\newline
\textbf{Fazer o elo entre usuário e máquina:} com o intuito de fazer isso de forma mais humanizada possível e também visamos o aconchego do usuário.\citep{site1}
\\\\
\begin{figure}[!htb]
     \centering
     \includegraphics[scale=0.3]{try2.jpg}
     \caption{"Comparação entre humanos e máquinas e humanização das mesmas."}
     \label{fig:imagem1}
\end{figure}
\newpage
\textbf{Entender e criar o elo entre Usuário e Computador:} um bom programador sabe como aperfeiçoar o seu código para ficar mais coerente e rápido em execução, um criador da interface que fará o elo entre usuário e máquina tem que saber com otimizar o mesmo, sem deixar coisas ambiguas ou que possam dificultar o entendimento da mensagem passada. Além que, ele terá que ter um bom conhecimento sobre o ser humano em si, saber que tipo de cores usar em determinadas situações e entre outras coisas que abragem o ser humano em si, não a máquina.\citep{site2}
\section{Relevância}
\paragraph{}Esta cadeira para um programador não chega a ser a mais importante, mas também não sai perdendo neste quesito, isto irá variar de acordo com a área que ele atuar. Porém, temos em mente que graças a esta cadeira ele poderá ter um controle mais abrangente sobre a situação de seu trabalho, além de melhorar o trabalho em grupo e o ajudará bastante na hora de desenvolver códigos ou sistemas que interajam com o usuário diretamente.
\section{Relação com outras disciplinas}
\begin{table}[h]
    \caption{Tabela de relação}
    \centering
    \begin{tabular}{| l | l |}
         \hline
         Disciplina & Relaçãos \\
        \hline
        IF793 - Projeto implementação de jogos 2D. & -Modo com que o jogador irá manipular\\& as ações do jogo.\\\hline
        IF682 - Engenharia de Software e Sistemas. & -Irá servir de elo entre o programa\\& em sí e a pessoa. \\\hline
        IF679 - Informática e Sociedade. & -Servirá para construir a apróximção \\&entre humano e máquina na sociedade.\\\hline
    \end{tabular}
    \label{tab:my_label}
\end{table}
\bibliographystyle{plain}
\bibliography{lvs2}
\end{document}

