\documentclass{article}
\usepackage[utf8]{inputenc}
\usepackage{url}
\title{IF702 - Redes Neurais}
\author{Gustavo Chaves Galdino de Moraes}
\date{04/05/2019}

\usepackage{natbib}
\usepackage{graphicx}

\begin{document}

\maketitle

\section{Introdução}
Redes neurais são sistemas de computação com nós interconectados que funcionam como os neurônios do cérebro humano. Usando algoritmos, elas podem reconhecer padrões escondidos e correlações em dados brutos, agrupá-los e classificá-los, e, com o tempo, aprender e melhorar continuamente.\citep{site2}
\newline
Os principais tópicos cobertos nesta disciplina são os principais fundamentos e modelos de Redes Neurais, aplicações e desenvolvimento de soluções.\citep{disciplina}
\newline
A disciplina de Redes Neurais se insere na grande área de Inteligência Artificial.
\begin{figure}[h!]
\centering
\includegraphics[scale=0.8]{imagem.png}\citep{wiki:1}
\caption{Esquema de uma Rede Neural simples}
\label{fig:download}
\end{figure}


\section{Relevância}    

O objetivo desta disciplina é fornecer ao aluno bagagem teórica e prática suficiente para que o mesmo consiga desenvolver soluções de problemas complexos em diversas situações da vida real utilizando o conhecimento de Redes neurais.\newline
Por isso, essa disciplina é tão relevante para a graduação em Ciencia da computação, pois ela permite o aluno uma ferramenta poderosa para solucionar problemas da vida real.
\section{Disciplinas relacionadas}
\begin{table}[h]
\begin{tabular}{lllll}
\cline{1-2}
\multicolumn{1}{|c|}{IF699 - Aprendizagem de Máquina} & \multicolumn{1}{l|}{\begin{tabular}[c]{@{}l@{}}Nessa disciplina, o aluno aprende métodos\\ e algoritmos que obtém conhecimento\\ a partir da análise de bases de dados,\\ uma habilidade essencial para\\ implementação de Redes Neurais.\end{tabular}} &  &  &  \\ \cline{1-2}
\multicolumn{1}{|l|}{IF684 - Sistemas Inteligentes} & \multicolumn{1}{l|}{\begin{tabular}[c]{@{}l@{}}Nessa disciplina, o aluno aprende\\  técnicas estatísticas para análise\\  de dados e de resultados\\  de modelos de simulação, o que está\\ intimamente relacionado com a disciplina\\ de Redes Neurais.\end{tabular}} &  &  &  \\ \cline{1-2}
 &  &  &  &  \\
 &  &  &  & 
\end{tabular}
\end{table}
\bibliographystyle{plain}
\bibliography{gcgm}

\end{document}
