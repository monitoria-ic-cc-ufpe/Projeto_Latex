\documentclass[10pt]{article}
\usepackage[utf8]{inputenc}

\title{IF702 - Redes Neurais}
\author{Pedro Sávio Silva Flor }
\date{Abril 2019}

\usepackage{natbib}
\usepackage{graphicx}
\usepackage{lipsum}
\usepackage{indentfirst}
\usepackage[portuguese]{babel}

\begin{document}

\maketitle

\section{Introdução}

O principal objetivo da cadeira em questão é apresentar a filosofia, os principais fundamentos, os
modelos de Redes Neurais, além de inserir os elementos supracitados em  aplicações e desenvolvimento de soluções.\citep{perfilcurricular}

Para realizar tal tarefa, o curso se divide em quatro partes. No primeiro segmento, trabalham-se os tópicos iniciais que apresentam uma introdução a redes neurais, seus fundamentos matemáticos, também se inclui nessa sessão o aprendizado sobre os fudamentos de modelos de aprendizagem. A segunda parte é sobre Arquiteturas e Modelos na qual se destrincha sobre: Redes feedforward, Redes recorrentes, Redes auto-organizáveis e Redes construtivas. A terceira parte se dedica a explorar o  desenvolvimento de soluções de problemas e aplicações. Por fim, a quarta parte é composta de um projeto em aplicação do mundo real. \citep{cin.ufpe}

As redes neurais estão principalmente inseridas no contexto do campo de inteligência artificial (IA), mais especificamente, na área de IA conexionista, a qual simula a estrutura do cérebro e se baseia na premissa de que a inteligência está na forma de processar informação e não na informação processada. \citep{tecmundo}


\begin{figure}[h!]
\centering
\includegraphics[scale=0.5]{Neuralnetwork1.png}
\caption{Exemplo de Rede Neural Artificial \citep{Imagem}}
\label{fig:universe}
\end{figure}

\section{Relevância}
Originalmente, os estudos relativos às redes neurais possuíam o intuito de realizar a gênese de um sistema computacional que trabalhasse para resolução de problemas similiramente ao cérebro humano. Ao longo do tempo, os pesquisadores superaram essa abordagem biológica e atualmente focam suas pesquisas em tarefas específicas. Nesse sentido, a importância da inclusão dessa cadeira sobre redes neurais é o esclarescimento sobre as possíveis aplicabilidades dessas para resoluções de diversas problemáticas.

Seguem alguns exemplos de áreas que o uso de redes neurais pode ser benéfico: Detecção de fraude em cartões de crédito e assistência médica; Diagnósticos médicos; Predições financeiras de ações de mercado, moeda, opções, futuros, falência e classificação de títulos.

\section{Relação com outras disciplinas}

\begin{table}[h]
 \centering
 {\renewcommand\arraystretch{1.25}
 \caption{ }
 \begin{tabular}{ l l }
  \cline{1-1}\cline{2-2}  
    \multicolumn{1}{|p{3.850cm}|}{Código e nome da discplina \citep{disciplinas}   \centering } &
    \multicolumn{1}{p{4.217cm}|}{Relação (Potencial uso dos conhecimentos obtidos na cadeira) \centering }
  \\  
  \cline{1-1}\cline{2-2}  
    \multicolumn{1}{|p{3.850cm}|}{IF752 - Análise imag. visão computacional} &
    \multicolumn{1}{p{4.217cm}|}{ Interpretação de fotos e vídeos não-tratados (por exemplo, na obtenção de imagens médicas)}
  \\  
  \cline{1-1}\cline{2-2}  
    \multicolumn{1}{|p{3.850cm}|}{IF798 - Robótica} &
    \multicolumn{1}{p{4.217cm}|}{Sistemas de controle robóticos}
  \\  
  \hline

 \end{tabular} }
\end{table}


\bibliographystyle{plain}
\bibliography{pssf}
\end{document}
