\documentclass{article}
\usepackage[utf8]{inputenc}

\title{IF708 - Programação Funcional}
\author{Renato Henrique Alpes Sampaio}
\date{}
\usepackage{natbib}
\usepackage{graphicx}

\begin{document}

\maketitle

\section{Introdução}
A disciplina Programação Funcional aborda o paradigma funcional de programação, atráves do uso de Haskell. "Em Haskell, nós tiramos a ênfase em código que modifica dados. Em lugar disso, focamos em funções que tomam valores imutáveis como entrada e produzem novos valores como saída. Dadas as mesmas entradas, essas funções sempre retornam os mesmos valores. Isto é uma idéia central na programação funcional."\citep{bryan2008realworldhaskell}. Além disso, a disciplina também aborda tópicos de lógica e algoritmos sob a visão da programação funcional. "Os principais tópicos estudados são: programação com funções, programação com listas, recursão, prova de propriedades sobre programas, provas por indução, inferência de tipos, tipos de dados algébricos, árvores, avaliação estrita e preguiçosa."\citep{homepage} Ao final do curso, terão sido estudadas também outras linguagens funcionais como F\# e Erlang.


\begin{figure}[h!]
\centering
\includegraphics[scale=0.85]{aaaa.png}
\caption{Função que calcula os 10 primeiros números da sequência de Fibonacci em Haskell.\citep{fibonacci}}
\end{figure}
\section{Relevância}
O uso de linguagens de programação funcional facilita a manutenção do código através de soluções concisas e elegantes, facilita o reuso de código atráves de sua modularidade e facilita a verificação por sua ausência de estado. Além disso, o estudo da programação funcional traz uma visão mais clara de conceitos fundamentais da computação, tais como funções, tipos de dados abstratos, overloading e polimorfismo. Ainda é possível observar também um interesse crescente na indústria, principalmente em sistemas em paralelo.\citep{slideintroducao}
\section{Relação com outras disciplinas}
\begin{table}[hbt!][]
\begin{tabular}{|l|l|}
\hline
\begin{tabular}[c]{@{}l@{}}IF686 - Paradigmas\\ de Linguagens Computacionais\end{tabular} & \begin{tabular}[c]{@{}l@{}}É pré-requisito\citep{perfilcurricularcc}, pois os conceitos \\ básicos de paradigma funcional \\ são introduzidos nesta disciplina.\end{tabular} \\ \hline
\end{tabular}
\end{table}
\bibliographystyle{plain}
\bibliography{rhas}
\end{document}
