\documentclass{article}
\usepackage[utf8]{inputenc}

\title{IF768 - Teoria de Grafos}
\author{Pedro Didier Maranhão}
\usepackage[brazil]{babel}
\date{April 2019}

\usepackage{natbib}
\usepackage{graphicx}

\begin{document}

\maketitle

\section{Introdução}
Teoria dos Grafos está inserida na grande área da lógica e se relaciona diretamente com o pensamento computacional. Dentro dessa disciplina é aprofundado o estudo de algumas relações lógicas as quais podem ser expressas na forma de vérticies e elos para facilitar sua visualização e interação. Alguns tópicos da disciplina são: \newline \newline
\textbf{Circuitos Eulerianos:} esses circuitos lógicos são a expressão direta dos grafos e são de vital importância para o entendimento da disciplina. \citep{referenciasdisciplina} \newline
\textbf{Busca em Profundidade:} esse tópico é de suma relevância para o cientista da computação, uma vez que se relaciona com a absorção de informação. Nesse caso, o assunto visa a simplificação matemática da busca em circuitos lógicos. \citep{referenciasdisciplina2} \newline
\textbf{Pontos de articulação:} fala sobre as várias formas de se espalhar informação por meio de um circuito conectado, focando na eficiência e na lógica para possibilitar soluções mais objetivas. \citep{referenciasdisciplina3} \newline
\textbf{Menor Caminho:} diz respeito às maneiras mais rápidas de percorrer circuitos lógicos, assim como o tópico anterior, também possui um grande foco na eficiência computacional. \citep{referenciasdisciplina4} \newline
\textbf{Fluxo Máximo:} foca na relação conexa do fluxo de informação ao longo de redes, tendo como base os circuitos lógicos, aplicando um maior foco, no entanto, na computação propriamente dita. \citep{referenciasdisciplina5}

\begin{figure}[h!]
\centering
\includegraphics[scale=0.5]{grafopqp.jpg}
\caption{Exemplo de Grafo}
\label{fig:grafolol}
\end{figure}

\section{Relevância}
A relevância de Teoria dos Grafos para um programador é ampla. A disciplina se relaciona às funções recursivas e diretas em linguagens de programação, à execução de algorítmos de busca e ordenação, dentre outras coisas. É possível também enxergar a necessidade do conhecimento de grafos para a área de sistemas de rede, uma vez que alguns tópicos da disciplina tratam diretamente do assunto. Fora isso o exercício lógico trazido pelo estudo dos grafos é de suma importância para a absorção do conhecimento computacional como um todo, assim como suas mais específicas minúncias.

\section{Relação Com Outras Disciplinas}
\begin{tabular}{c|c}
  Disciplina   &  Relações \\
   & \\
  IF670 - Matemática Discreta Para Computação  & Tem estudo de Grafos e base na lógica. \\
  IF673 - Lógica para Computação & Teoria dos Grafos é uma sub-área da lógica. \\
  IF672 - Algoritmos e Estruturas De Dados & Uso de grafos para optimizar programação.	
\end{tabular}

\bibliographystyle{plain}
\bibliography{references}
\end{document}
