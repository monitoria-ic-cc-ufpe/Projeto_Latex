\documentclass{article}
\usepackage[utf8]{inputenc}

\title{IF796 - Mineração da Web}
\author{Carlos Augusto Mendes Valença}
\date{May 2019}

\usepackage{natbib}
\usepackage{graphicx}

\begin{document}

\maketitle

\section{Introdução}
A Mineração da Web é uma disciplina da área de Recuperação de Informação (RI) que trabalha com obtenção e organização de informações, documentos e outras formas de organização de dados, assim como trabalhar com bases de índices. 

\section{Relevância}
A disciplina é importante graças à sua aplicação em diversas áreas da Computação relacionadas à manejo de informações, como, por exemplo sistemas da Web como engenhos de busca e sistemas de recomendações a usuários, a exemplo do sistema do Youtube, assim como ferramentas do tipo Page Rank. 
\subsection{Positivos}
\item Útil para várias áreas relacionadas ao uso de dados.


\begin{figure}[h!]
\centering
\includegraphics[scale=1.7]{universe}
\caption{The Universe}
\label{fig:universe}
\end{figure}

\section{Conclusion}
``I always thought something was fundamentally wrong with the universe'' \citep{adams1995hitchhiker}

\bibliographystyle{plain}
\bibliography{references}
\end{document}
