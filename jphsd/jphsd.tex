\documentclass{article}
\usepackage[utf8]{inputenc}

\title{IF678 - Infra-estrutura de Comunicação}
\author{João Pedro Henrique}
\date{April 2019}

\usepackage{natbib}
\usepackage{graphicx}
\usepackage{float}
\restylefloat{table}

\begin{document}

\maketitle

\section{Introdução}
\includegraphics[scale=0.42]{redes.jpg}

\textbf{Infra-estrutura de Comunicação} é uma disciplina do terceiro período do curso de Ciência da Computação. Essa disciplina visa introduzir o aluno a diversos conceitos das redes de computadores, tais como sua história, as camadas envolvidas na construção de sistemas de comunicação, a internet, assim como os protocolos que visam padronizar o comportamento destas redes ao redor do mundo.

\section{Relevância}
Em um mundo cada vez mais conectado, compreender o funcionamento dos sistemas de comunicação entre máquinas torna-se algo crucial para futuros cientistas da computação.

Sendo assim, o conhecimento de redes mostra-se um conhecimento tão básico quanto o domínio das técnicas de Hardware e Software, permitindo que o egresso do curso possa construir sistemas que utilizem as melhores arquiteturas nos três pontos básicos.

\section{Relação com outras disciplinas}

Como a cadeira visa apresentar ao aluno um fundamento da computação moderna, é esperado que existam relações com diversas outras disciplinas da Matriz Curricular do curso.

\begin{table}[]
\begin{tabular}{|l|l|}
\hline
Código                                                                      & Relação \\ \hline
\begin{tabular}[c]{@{}l@{}}IF741\\ Gerenciamento\\ de Redes\end{tabular}    & \begin{tabular}[c]{@{}l@{}}Por essa ser uma disciplina que visa ensinar o\\ gerenciamento de redes, é crucial que o aluno tenha o\\ conhecimento de como funcionam as redes de computadores\\ e seus principais protocolos, que são tópicos da disciplina\\ de Infra-estrutura de comunicação.\end{tabular}                                     \\ \hline
\begin{tabular}[c]{@{}l@{}}IF712\\ Programação\\ para Internet\end{tabular} & \begin{tabular}[c]{@{}l@{}}Por ser uma disciplina com foco em introduzir o aluno às\\ técnicas de produção de software para a internet, é crucial\\ que o aluno compreenda os protocolos de comunicação\\ que regem a internet.\end{tabular}                                                                                                    \\ \hline
\begin{tabular}[c]{@{}l@{}}IF740\\ Sistemas de\\ Comunicação\end{tabular}   & \begin{tabular}[c]{@{}l@{}}Essa disciplina está focada em introduzir o aluno à Sistemas\\ Gerenciadores de Redes. Todavia, para isso é necessário que\\ o aluno possua conhecimento da infra-estrutura para qual\\ está construindo essas interfaces, conhecimento adquirido\\ na disciplina de de Infra-estrutura de comunicação.\end{tabular} \\ \hline
\end{tabular}
\end{table}

\section{Referências}
A disciplina utiliza como base o livro \textbf{Redes de Computadores e a Internet: uma abordagem top-down} tanto em sua sexta \cite{kurose6}, quanto quinta edição \cite{kurose5}. Além disso, é utilizado, como bibliografia adicional, o livro \textbf{Redes de Computadores} \cite{tanenbaum}.

\bibliographystyle{plain}
\bibliography{jphsd}
\end{document}
