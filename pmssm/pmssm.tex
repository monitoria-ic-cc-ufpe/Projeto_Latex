\documentclass[10pt]{article}
\usepackage[utf8]{inputenc}
\usepackage[portuguese]{babel}

\title{IF675 - Sistemas Digitais}
\author{Paulo Matheus}
\date{May 2019}

\usepackage{natbib}
\usepackage{graphicx}

\begin{document}

\maketitle

\section{Introdução}
O curso de Sistemas Digitais tem uma carga horária de 75 horas, e tem como objetivo dar ao aluno conhecimentos de circuitos lógicos digitais combinacionais e sequências cobrindo desde dispositivos digitais de pequena complexidade SSI, até a implementação de circuitos de média complexidade MSI.
O curso é ministrado pelo Professor Manoel Eusebio de Lima,  professor do Centro de Informática da UFPE.
\cite{intro}


\begin{figure}[h!]
\centering
\includegraphics[scale=0.60]{sistemas.jpg}
\caption{Processador}
\label{fig:processador}
\end{figure}


\section{Relevância}
Com base dos ensinamentos desta cadeira conseguimos construir circuitos lógicos, como processadores, essências na computação.

\section{Relação}

\begin{center}
\begin{tabular}{|c|p{6cm}|}
\hline
códigos & relações \\ \hline
 IF674- Infra-estrutura de hardware &
 O Curso de Infra-Estrutura de Hardware visa dar uma visão geral dos componentes de um computador, quais sejam: processador, sistema de memória (memória principal e memória cache), Entrada e Saída e Barramentos. Nesta disciplina os princípios de funcionamentos de cada um dos componentes acima serão apresentados e o aluno terá possibilidade de sedimentar estes conceitos seja pelo projeto de uma versão simples do componente, seja pela simulação do mesmo através de ferramenta de simulação. \cite{IF674}
 \\ \hline
  IF687 - Introdução à Multimídia  & 
Cadeira que visa introduzir o aluno a lógica de programação, e se relaciona com a cadeira de sistemas digitais pois o aluno precisa de uma base lógica para a mesma. \cite{IF669}
\\ \hline

\end{tabular}    
\end{center}

\bibliographystyle{plain}
\bibliography{pmssm}
\end{document}